%
\begin{isabellebody}%
\setisabellecontext{Ordered{\isacharunderscore}Trail}%
%
\isadelimtheory
%
\endisadelimtheory
%
\isatagtheory
%
\endisatagtheory
{\isafoldtheory}%
%
\isadelimtheory
%
\endisadelimtheory
%
\isadelimdocument
%
\endisadelimdocument
%
\isatagdocument
%
\isamarkupsection{Graph Theory in the Archive of Formal Proofs%
}
\isamarkuptrue%
%
\endisatagdocument
{\isafolddocument}%
%
\isadelimdocument
%
\endisadelimdocument
%
\begin{isamarkuptext}%
\label{GraphTheory} To increase the reusability of our library we build upon the \isa{Graph{\isacharunderscore}Theory}
library by Noschinski \cite{Graph_Theory-AFP}. Graphs are represented as records consisting of vertices and edges that
can be accessed using the selectors \isa{pverts} and \isa{parcs}. We recall the definition 
of the type \isa{pair{\isacharunderscore}pre{\isacharunderscore}digraph}:

 %
\begin{isabelle}%
record\ {\isacharprime}a\ pair{\isacharunderscore}pre{\isacharunderscore}digraph\ {\isacharequal}\ pverts\ {\isacharcolon}{\isacharcolon}\ {\isachardoublequote}{\isacharprime}a\ set{\isachardoublequote}\ parcs\ {\isacharcolon}{\isacharcolon}\ {\isachardoublequote}{\isacharprime}a\ rel{\isachardoublequote}%
\end{isabelle}

Now restrictions upon the two sets and new features can be introduced using locales. 
Locales are Isabelle's way to deal with parameterized theories~\cite{ballarin2010tutorial}. Consider
for example \isa{pair{\isacharunderscore}wf{\isacharunderscore}digraph}.
The endpoints of an edge can be accessed using the functions \isa{fst} and \isa{snd}. Therefore, conditions
\isa{arc{\isacharunderscore}fst{\isacharunderscore}in{\isacharunderscore}verts} and \isa{arc{\isacharunderscore}snd{\isacharunderscore}in{\isacharunderscore}verts} assert that both endpoints of an edge are
vertices. Using so-called sublocales a variety of other graphs are defined.  

%
\begin{isabelle}%
locale\ pair{\isacharunderscore}wf{\isacharunderscore}digraph\ {\isacharequal}\ pair{\isacharunderscore}pre{\isacharunderscore}digraph\ {\isacharplus}\isanewline
\ \ assumes\ arc{\isacharunderscore}fst{\isacharunderscore}in{\isacharunderscore}verts{\isacharcolon}\ {\isachardoublequote}{\isasymAnd}e{\isachardot}\ e\ {\isasymin}\ parcs\ G\ {\isasymLongrightarrow}\ fst\ e\ {\isasymin}\ pverts\ G{\isachardoublequote}\isanewline
\ \ assumes\ arc{\isacharunderscore}snd{\isacharunderscore}in{\isacharunderscore}verts{\isacharcolon}\ {\isachardoublequote}{\isasymAnd}e{\isachardot}\ e\ {\isasymin}\ parcs\ G\ {\isasymLongrightarrow}\ snd\ e\ {\isasymin}\ pverts\ G{\isachardoublequote}%
\end{isabelle}

An object of type \isa{{\isacharprime}b\ awalk} is defined in \isa{Graph{\isacharunderscore}Theory{\isachardot}Arc{\isacharunderscore}Walk} as a list of edges. 
Additionally, the definition \isa{awalk} imposes that both endpoints of a walk are vertices of 
the graph, all elements of the walk are edges and two subsequent edges share a common vertex. \vspace{1em}

\noindent\isa{type{\isacharunderscore}synonym\ {\isacharprime}b\ awalk\ {\isacharequal}\ {\isachardoublequote}{\isacharprime}b\ list{\isachardoublequote}}

%
\begin{isabelle}%
definition\ awalk\ {\isacharcolon}{\isacharcolon}\ {\isachardoublequote}{\isacharprime}a\ {\isasymRightarrow}\ {\isacharprime}b\ awalk\ {\isasymRightarrow}\ {\isacharprime}a\ {\isasymRightarrow}\ bool{\isachardoublequote}\isanewline
{\isachardoublequote}awalk\ u\ p\ v\ {\isasymequiv}\ u\ {\isasymin}\ verts\ G\ {\isasymand}\ set\ p\ {\isasymsubseteq}\ arcs\ G\ {\isasymand}\ cas\ u\ p\ v{\isachardoublequote}\ %
\end{isabelle}

\noindent We also reuse the type synonym \isa{weight{\isacharunderscore}fun} introduced in \mbox{\isa{Weighted{\isacharunderscore}Graph}}. \vspace{1em}

\isa{type{\isacharunderscore}synonym\ {\isacharprime}b\ weight{\isacharunderscore}fun\ {\isacharequal}\ {\isachardoublequote}{\isacharprime}b\ {\isasymRightarrow}\ real{\isachardoublequote}}  \vspace{1em}

Finally, there is an useful definition capturing the notion of a complete graph, namely \isa{complete{\isacharunderscore}digraph}.%
\end{isamarkuptext}\isamarkuptrue%
%
\isadelimdocument
%
\endisadelimdocument
%
\isatagdocument
%
\isamarkupsection{Formalization of Trail Properties in Isabelle/HOL%
}
\isamarkuptrue%
%
\endisatagdocument
{\isafolddocument}%
%
\isadelimdocument
%
\endisadelimdocument
%
\begin{isamarkuptext}%
\label{Formalization}%
\end{isamarkuptext}\isamarkuptrue%
%
\isadelimdocument
%
\endisadelimdocument
%
\isatagdocument
%
\isamarkupsubsection{Increasing and Decreasing Trails in Weighted Graphs%
}
\isamarkuptrue%
%
\endisatagdocument
{\isafolddocument}%
%
\isadelimdocument
%
\endisadelimdocument
%
\begin{isamarkuptext}%
\label{trails} In our work we extend the graph theory framework from Section \ref{GraphTheory} 
with new features enabling reasoning about trails. To this end,
 a trail is defined as a list of edges. We will only consider strictly increasing trails 
on graphs without parallel edges. For this we require the graph 
to be of type \isa{pair{\isacharunderscore}pre{\isacharunderscore}digraph}, as introduced in Section \ref{GraphTheory}. 

Two different definitions
are given in our formalization. Function \isa{incTrail} can be used without specifying the first and last vertex of the trail
whereas \isa{incTrail{\isadigit{2}}} uses more of \isa{Graph{\isacharunderscore}Theory{\isacharprime}s} predefined features. Moreover, making use of  monotonicity
\mbox{\isa{incTrail}} only requires to check if one edge's weight is smaller than its successors' while \isa{incTrail{\isadigit{2}}}
checks if the weight is smaller than the one of all subsequent edges in the sequence, i.e. if the list 
is sorted. The {\em equivalence
between the two notions} is shown in the following.%
\end{isamarkuptext}\isamarkuptrue%
\isacommand{fun}\isamarkupfalse%
\ incTrail\ {\isacharcolon}{\isacharcolon}\ {\isachardoublequoteopen}{\isacharprime}a\ pair{\isacharunderscore}pre{\isacharunderscore}digraph\ {\isasymRightarrow}\ {\isacharparenleft}{\isacharprime}a\ {\isasymtimes}{\isacharprime}a{\isacharparenright}\ weight{\isacharunderscore}fun\ {\isasymRightarrow}\ {\isacharparenleft}{\isacharprime}a\ {\isasymtimes}{\isacharprime}a{\isacharparenright}\ list\ {\isasymRightarrow}\ bool{\isachardoublequoteclose}\ \isakeyword{where}\isanewline
{\isachardoublequoteopen}incTrail\ g\ w\ {\isacharbrackleft}{\isacharbrackright}\ {\isacharequal}\ True{\isachardoublequoteclose}\ {\isacharbar}\isanewline
{\isachardoublequoteopen}incTrail\ g\ w\ {\isacharbrackleft}e\isactrlsub {\isadigit{1}}{\isacharbrackright}\ {\isacharequal}\ {\isacharparenleft}e\isactrlsub {\isadigit{1}}\ {\isasymin}\ parcs\ g{\isacharparenright}{\isachardoublequoteclose}\ {\isacharbar}\isanewline
{\isachardoublequoteopen}incTrail\ g\ w\ {\isacharparenleft}e\isactrlsub {\isadigit{1}}{\isacharhash}e\isactrlsub {\isadigit{2}}{\isacharhash}es{\isacharparenright}\ {\isacharequal}\ {\isacharparenleft}if\ w\ e\isactrlsub {\isadigit{1}}\ {\isacharless}\ w\ e\isactrlsub {\isadigit{2}}\ {\isasymand}\ e\isactrlsub {\isadigit{1}}\ {\isasymin}\ parcs\ g\ {\isasymand}\ snd\ e\isactrlsub {\isadigit{1}}\ {\isacharequal}\ fst\ e\isactrlsub {\isadigit{2}}\ \isanewline
\ \ \ \ \ \ \ \ \ \ \ \ \ \ \ \ \ \ \ \ \ \ \ \ \ \ \ \ \ \ \ \ \ \ \ \ then\ incTrail\ g\ w\ {\isacharparenleft}e\isactrlsub {\isadigit{2}}{\isacharhash}es{\isacharparenright}\ else\ False{\isacharparenright}{\isachardoublequoteclose}\isanewline
\isanewline
\isacommand{definition}\isamarkupfalse%
{\isacharparenleft}\isakeyword{in}\ pair{\isacharunderscore}pre{\isacharunderscore}digraph{\isacharparenright}\ incTrail{\isadigit{2}}\ \isakeyword{where}\ \isanewline
{\isachardoublequoteopen}incTrail{\isadigit{2}}\ w\ es\ u\ v\ {\isasymequiv}\ sorted{\isacharunderscore}wrt\ {\isacharparenleft}{\isasymlambda}\ e\isactrlsub {\isadigit{1}}\ e\isactrlsub {\isadigit{2}}{\isachardot}\ w\ e\isactrlsub {\isadigit{1}}\ {\isacharless}\ w\ e\isactrlsub {\isadigit{2}}{\isacharparenright}\ es\ {\isasymand}\ {\isacharparenleft}es\ {\isacharequal}\ {\isacharbrackleft}{\isacharbrackright}\ {\isasymor}\ awalk\ u\ es\ v{\isacharparenright}{\isachardoublequoteclose}\isanewline
\isanewline
\isanewline
\isanewline
\isacommand{fun}\isamarkupfalse%
\ decTrail\ {\isacharcolon}{\isacharcolon}\ {\isachardoublequoteopen}{\isacharprime}a\ pair{\isacharunderscore}pre{\isacharunderscore}digraph\ {\isasymRightarrow}\ {\isacharparenleft}{\isacharprime}a\ {\isasymtimes}{\isacharprime}a{\isacharparenright}\ weight{\isacharunderscore}fun\ {\isasymRightarrow}\ {\isacharparenleft}{\isacharprime}a\ {\isasymtimes}{\isacharprime}a{\isacharparenright}\ list\ {\isasymRightarrow}\ bool{\isachardoublequoteclose}\ \isakeyword{where}\isanewline
{\isachardoublequoteopen}decTrail\ g\ w\ {\isacharbrackleft}{\isacharbrackright}\ {\isacharequal}\ True{\isachardoublequoteclose}\ {\isacharbar}\isanewline
{\isachardoublequoteopen}decTrail\ g\ w\ {\isacharbrackleft}e\isactrlsub {\isadigit{1}}{\isacharbrackright}\ {\isacharequal}\ {\isacharparenleft}e\isactrlsub {\isadigit{1}}\ {\isasymin}\ parcs\ g{\isacharparenright}{\isachardoublequoteclose}\ {\isacharbar}\isanewline
{\isachardoublequoteopen}decTrail\ g\ w\ {\isacharparenleft}e\isactrlsub {\isadigit{1}}{\isacharhash}e\isactrlsub {\isadigit{2}}{\isacharhash}es{\isacharparenright}\ {\isacharequal}\ {\isacharparenleft}if\ w\ e\isactrlsub {\isadigit{1}}\ {\isachargreater}\ w\ e\isactrlsub {\isadigit{2}}\ {\isasymand}\ e\isactrlsub {\isadigit{1}}\ {\isasymin}\ parcs\ g\ {\isasymand}\ snd\ e\isactrlsub {\isadigit{1}}\ {\isacharequal}\ fst\ e\isactrlsub {\isadigit{2}}\ \isanewline
\ \ \ \ \ \ \ \ \ \ \ \ \ \ \ \ \ \ \ \ \ \ \ \ \ \ \ \ \ \ \ \ \ \ \ \ then\ decTrail\ g\ w\ {\isacharparenleft}e\isactrlsub {\isadigit{2}}{\isacharhash}es{\isacharparenright}\ else\ False{\isacharparenright}{\isachardoublequoteclose}\isanewline
\isanewline
\isacommand{definition}\isamarkupfalse%
{\isacharparenleft}\isakeyword{in}\ pair{\isacharunderscore}pre{\isacharunderscore}digraph{\isacharparenright}\ decTrail{\isadigit{2}}\ \isakeyword{where}\ \isanewline
{\isachardoublequoteopen}decTrail{\isadigit{2}}\ w\ es\ u\ v\ {\isasymequiv}\ sorted{\isacharunderscore}wrt\ {\isacharparenleft}{\isasymlambda}\ e\isactrlsub {\isadigit{1}}\ e\isactrlsub {\isadigit{2}}{\isachardot}\ w\ e\isactrlsub {\isadigit{1}}\ {\isachargreater}\ w\ e\isactrlsub {\isadigit{2}}{\isacharparenright}\ es\ {\isasymand}\ {\isacharparenleft}es\ {\isacharequal}\ {\isacharbrackleft}{\isacharbrackright}\ {\isasymor}\ awalk\ u\ es\ v{\isacharparenright}{\isachardoublequoteclose}%
\begin{isamarkuptext}%
Defining trails as lists in Isabelle has many advantages including using predefined list operators, 
e.g., drop. Thus, we can show one result that will be constantly needed in the following, that is, that
{\em any subtrail of an ordered trail is an ordered trail itself}.%
\end{isamarkuptext}\isamarkuptrue%
\isacommand{lemma}\isamarkupfalse%
\ incTrail{\isacharunderscore}subtrail{\isacharcolon}\ \isanewline
\ \ \isakeyword{assumes}\ {\isachardoublequoteopen}incTrail\ g\ w\ es{\isachardoublequoteclose}\isanewline
\ \ \isakeyword{shows}\ {\isachardoublequoteopen}incTrail\ g\ w\ {\isacharparenleft}drop\ k\ es{\isacharparenright}{\isachardoublequoteclose}%
\isadelimproof
%
\endisadelimproof
%
\isatagproof
%
\endisatagproof
{\isafoldproof}%
%
\isadelimproof
%
\endisadelimproof
%
\begin{isamarkuptext}%
%
\end{isamarkuptext}\isamarkuptrue%
\isacommand{lemma}\isamarkupfalse%
\ decTrail{\isacharunderscore}subtrail{\isacharcolon}\ \isanewline
\ \ \isakeyword{assumes}\ {\isachardoublequoteopen}decTrail\ g\ w\ es{\isachardoublequoteclose}\isanewline
\ \ \isakeyword{shows}\ {\isachardoublequoteopen}decTrail\ g\ w\ {\isacharparenleft}drop\ k\ es{\isacharparenright}{\isachardoublequoteclose}%
\isadelimproof
%
\endisadelimproof
%
\isatagproof
%
\endisatagproof
{\isafoldproof}%
%
\isadelimproof
%
\endisadelimproof
%
\isadelimproof
%
\endisadelimproof
%
\isatagproof
%
\endisatagproof
{\isafoldproof}%
%
\isadelimproof
%
\endisadelimproof
%
\isadelimproof
%
\endisadelimproof
%
\isatagproof
%
\endisatagproof
{\isafoldproof}%
%
\isadelimproof
%
\endisadelimproof
%
\isadelimproof
%
\endisadelimproof
%
\isatagproof
%
\endisatagproof
{\isafoldproof}%
%
\isadelimproof
%
\endisadelimproof
%
\isadelimproof
%
\endisadelimproof
%
\isatagproof
%
\endisatagproof
{\isafoldproof}%
%
\isadelimproof
%
\endisadelimproof
%
\isadelimproof
%
\endisadelimproof
%
\isatagproof
%
\endisatagproof
{\isafoldproof}%
%
\isadelimproof
%
\endisadelimproof
%
\isadelimproof
%
\endisadelimproof
%
\isatagproof
%
\endisatagproof
{\isafoldproof}%
%
\isadelimproof
%
\endisadelimproof
%
\isadelimproof
%
\endisadelimproof
%
\isatagproof
%
\endisatagproof
{\isafoldproof}%
%
\isadelimproof
%
\endisadelimproof
%
\isadelimproof
%
\endisadelimproof
%
\isatagproof
%
\endisatagproof
{\isafoldproof}%
%
\isadelimproof
%
\endisadelimproof
%
\isadelimproof
%
\endisadelimproof
%
\isatagproof
%
\endisatagproof
{\isafoldproof}%
%
\isadelimproof
%
\endisadelimproof
%
\isadelimproof
%
\endisadelimproof
%
\isatagproof
%
\endisatagproof
{\isafoldproof}%
%
\isadelimproof
%
\endisadelimproof
%
\begin{isamarkuptext}%
In Isabelle we then show the equivalence between the two definitions \isa{decTrail} and \isa{decTrail{\isadigit{2}}} of strictly decreasing trails.
Similarly, we also show the equivalence between the definition \isa{incTrail} and \isa{incTrail{\isadigit{2}}} of strictly increasing trails.%
\end{isamarkuptext}\isamarkuptrue%
\isacommand{lemma}\isamarkupfalse%
{\isacharparenleft}\isakeyword{in}\ pair{\isacharunderscore}wf{\isacharunderscore}digraph{\isacharparenright}\ decTrail{\isacharunderscore}is{\isacharunderscore}dec{\isacharunderscore}walk{\isacharcolon}\isanewline
\ \ \isakeyword{shows}\ {\isachardoublequoteopen}decTrail\ G\ w\ es\ {\isasymlongleftrightarrow}\ decTrail{\isadigit{2}}\ w\ es\ {\isacharparenleft}fst\ {\isacharparenleft}hd\ es{\isacharparenright}{\isacharparenright}\ {\isacharparenleft}snd\ {\isacharparenleft}last\ es{\isacharparenright}{\isacharparenright}{\isachardoublequoteclose}%
\isadelimproof
%
\endisadelimproof
%
\isatagproof
%
\endisatagproof
{\isafoldproof}%
%
\isadelimproof
%
\endisadelimproof
%
\begin{isamarkuptext}%
%
\end{isamarkuptext}\isamarkuptrue%
\isacommand{lemma}\isamarkupfalse%
{\isacharparenleft}\isakeyword{in}\ pair{\isacharunderscore}wf{\isacharunderscore}digraph{\isacharparenright}\ incTrail{\isacharunderscore}is{\isacharunderscore}inc{\isacharunderscore}walk{\isacharcolon}\isanewline
\ \ \isakeyword{shows}\ {\isachardoublequoteopen}incTrail\ G\ w\ es\ {\isasymlongleftrightarrow}\ incTrail{\isadigit{2}}\ w\ es\ {\isacharparenleft}fst\ {\isacharparenleft}hd\ es{\isacharparenright}{\isacharparenright}\ {\isacharparenleft}snd\ {\isacharparenleft}last\ es{\isacharparenright}{\isacharparenright}{\isachardoublequoteclose}%
\isadelimproof
%
\endisadelimproof
%
\isatagproof
%
\endisatagproof
{\isafoldproof}%
%
\isadelimproof
%
\endisadelimproof
%
\isadelimproof
%
\endisadelimproof
%
\isatagproof
%
\endisatagproof
{\isafoldproof}%
%
\isadelimproof
%
\endisadelimproof
%
\isadelimproof
%
\endisadelimproof
%
\isatagproof
%
\endisatagproof
{\isafoldproof}%
%
\isadelimproof
%
\endisadelimproof
%
\begin{isamarkuptext}%
Any strictly decreasing trail $(e_1,\ldots,e_n)$ can also be seen as a strictly increasing trail $(e_n,...,e_1)$
if the graph considered is undirected. To this end, we make use of the locale \isa{pair{\isacharunderscore}sym{\isacharunderscore}digraph}
that captures the idea of symmetric arcs. However, it is also necessary to assume that the weight 
function assigns the same weight to edge $(v_i,v_j)$ as to $(v_j,v_i)$. This assumption is therefore
added to \isa{decTrail{\isacharunderscore}eq{\isacharunderscore}rev{\isacharunderscore}incTrail} and \isa{incTrail{\isacharunderscore}eq{\isacharunderscore}rev{\isacharunderscore}decTrail}.%
\end{isamarkuptext}\isamarkuptrue%
\isacommand{lemma}\isamarkupfalse%
{\isacharparenleft}\isakeyword{in}\ pair{\isacharunderscore}sym{\isacharunderscore}digraph{\isacharparenright}\ decTrail{\isacharunderscore}eq{\isacharunderscore}rev{\isacharunderscore}incTrail{\isacharcolon}\isanewline
\ \ \isakeyword{assumes}\ {\isachardoublequoteopen}{\isasymforall}\ v\isactrlsub {\isadigit{1}}\ v\isactrlsub {\isadigit{2}}{\isachardot}\ w\ {\isacharparenleft}v\isactrlsub {\isadigit{1}}{\isacharcomma}v\isactrlsub {\isadigit{2}}{\isacharparenright}\ {\isacharequal}\ w{\isacharparenleft}v\isactrlsub {\isadigit{2}}{\isacharcomma}v\isactrlsub {\isadigit{1}}{\isacharparenright}{\isachardoublequoteclose}\ \isanewline
\ \ \isakeyword{shows}\ {\isachardoublequoteopen}decTrail\ G\ w\ es\ {\isasymlongleftrightarrow}\ incTrail\ G\ w\ {\isacharparenleft}rev\ {\isacharparenleft}map\ {\isacharparenleft}{\isasymlambda}{\isacharparenleft}v\isactrlsub {\isadigit{1}}{\isacharcomma}v\isactrlsub {\isadigit{2}}{\isacharparenright}{\isachardot}\ {\isacharparenleft}v\isactrlsub {\isadigit{2}}{\isacharcomma}v\isactrlsub {\isadigit{1}}{\isacharparenright}{\isacharparenright}\ es{\isacharparenright}{\isacharparenright}{\isachardoublequoteclose}\isanewline
\ \ \ \ %
\isadelimproof
%
\endisadelimproof
%
\isatagproof
%
\endisatagproof
{\isafoldproof}%
%
\isadelimproof
%
\endisadelimproof
\isanewline
\isacommand{lemma}\isamarkupfalse%
{\isacharparenleft}\isakeyword{in}\ pair{\isacharunderscore}sym{\isacharunderscore}digraph{\isacharparenright}\ incTrail{\isacharunderscore}eq{\isacharunderscore}rev{\isacharunderscore}decTrail{\isacharcolon}\isanewline
\ \ \isakeyword{assumes}\ {\isachardoublequoteopen}{\isasymforall}\ v\isactrlsub {\isadigit{1}}\ v\isactrlsub {\isadigit{2}}{\isachardot}\ w\ {\isacharparenleft}v\isactrlsub {\isadigit{1}}{\isacharcomma}v\isactrlsub {\isadigit{2}}{\isacharparenright}\ {\isacharequal}\ w{\isacharparenleft}v\isactrlsub {\isadigit{2}}{\isacharcomma}v\isactrlsub {\isadigit{1}}{\isacharparenright}{\isachardoublequoteclose}\ \isanewline
\ \ \isakeyword{shows}\ {\isachardoublequoteopen}incTrail\ G\ w\ es\ {\isasymlongleftrightarrow}\ decTrail\ G\ w\ {\isacharparenleft}rev\ {\isacharparenleft}map\ {\isacharparenleft}{\isasymlambda}{\isacharparenleft}v\isactrlsub {\isadigit{1}}{\isacharcomma}v\isactrlsub {\isadigit{2}}{\isacharparenright}{\isachardot}\ {\isacharparenleft}v\isactrlsub {\isadigit{2}}{\isacharcomma}v\isactrlsub {\isadigit{1}}{\isacharparenright}{\isacharparenright}\ es{\isacharparenright}{\isacharparenright}{\isachardoublequoteclose}%
\isadelimproof
%
\endisadelimproof
%
\isatagproof
%
\endisatagproof
{\isafoldproof}%
%
\isadelimproof
%
\endisadelimproof
%
\isadelimdocument
%
\endisadelimdocument
%
\isatagdocument
%
\isamarkupsubsection{Weighted Graphs%
}
\isamarkuptrue%
%
\endisatagdocument
{\isafolddocument}%
%
\isadelimdocument
%
\endisadelimdocument
%
\begin{isamarkuptext}%
\label{localeSurjective} We add the locale \isa{weighted{\isacharunderscore}pair{\isacharunderscore}graph} 
on top of the locale \isa{pair{\isacharunderscore}graph} introduced in \isa{Graph{\isacharunderscore}Theory}. A  \isa{pair{\isacharunderscore}graph} is a 
finite, loop free and symmetric graph. We do not restrict the types of vertices and edges but impose 
the condition that they have to be a linear order.

 Furthermore, all weights have to be integers between 0 and $\lfloor\frac{q}{2}\rfloor$ where 0 is 
used as a special value to indicate that there is no edge at that position. Since the 
range of the weight function is in the reals, the set of natural numbers
\mbox{\isa{{\isacharbraceleft}{\isadigit{1}}{\isacharcomma}{\isachardot}{\isachardot}{\isacharcomma}card\ {\isacharparenleft}parcs\ G{\isacharparenright}\ div\ {\isadigit{2}}{\isacharbraceright}}} has to be casted into a set of reals. This is realized by taking the image
of the function \isa{real} that casts a natural number to a real.%
\end{isamarkuptext}\isamarkuptrue%
\isacommand{locale}\isamarkupfalse%
\ weighted{\isacharunderscore}pair{\isacharunderscore}graph\ {\isacharequal}\ pair{\isacharunderscore}graph\ {\isachardoublequoteopen}{\isacharparenleft}G{\isacharcolon}{\isacharcolon}\ {\isacharparenleft}{\isacharprime}a{\isacharcolon}{\isacharcolon}linorder{\isacharparenright}\ pair{\isacharunderscore}pre{\isacharunderscore}digraph{\isacharparenright}{\isachardoublequoteclose}\ \isakeyword{for}\ G\ {\isacharplus}\isanewline
\ \ \isakeyword{fixes}\ w\ {\isacharcolon}{\isacharcolon}\ {\isachardoublequoteopen}{\isacharparenleft}{\isacharprime}a{\isasymtimes}{\isacharprime}a{\isacharparenright}\ weight{\isacharunderscore}fun{\isachardoublequoteclose}\isanewline
\ \ \isakeyword{assumes}\ dom{\isacharcolon}\ {\isachardoublequoteopen}e\ {\isasymin}\ parcs\ G\ {\isasymlongrightarrow}\ w\ e\ {\isasymin}\ real\ {\isacharbackquote}\ {\isacharbraceleft}{\isadigit{1}}{\isachardot}{\isachardot}card\ {\isacharparenleft}parcs\ G{\isacharparenright}\ div\ {\isadigit{2}}{\isacharbraceright}{\isachardoublequoteclose}\ \isanewline
\ \ \ \ \ \ \isakeyword{and}\ vert{\isacharunderscore}ge{\isacharcolon}\ {\isachardoublequoteopen}card\ {\isacharparenleft}pverts\ G{\isacharparenright}\ {\isasymge}\ {\isadigit{1}}{\isachardoublequoteclose}\ %
\begin{isamarkuptext}%
We introduce some useful abbreviations, according to the ones in Section \ref{Prelim}%
\end{isamarkuptext}\isamarkuptrue%
\isacommand{abbreviation}\isamarkupfalse%
{\isacharparenleft}\isakeyword{in}\ weighted{\isacharunderscore}pair{\isacharunderscore}graph{\isacharparenright}\ {\isachardoublequoteopen}q\ {\isasymequiv}\ card\ {\isacharparenleft}parcs\ G{\isacharparenright}{\isachardoublequoteclose}\isanewline
\isacommand{abbreviation}\isamarkupfalse%
{\isacharparenleft}\isakeyword{in}\ weighted{\isacharunderscore}pair{\isacharunderscore}graph{\isacharparenright}\ {\isachardoublequoteopen}n\ {\isasymequiv}\ card\ {\isacharparenleft}pverts\ G{\isacharparenright}{\isachardoublequoteclose}\isanewline
\isacommand{abbreviation}\isamarkupfalse%
{\isacharparenleft}\isakeyword{in}\ weighted{\isacharunderscore}pair{\isacharunderscore}graph{\isacharparenright}\ {\isachardoublequoteopen}W\ {\isasymequiv}\ {\isacharbraceleft}{\isadigit{1}}{\isachardot}{\isachardot}q\ div\ {\isadigit{2}}{\isacharbraceright}{\isachardoublequoteclose}%
\begin{isamarkuptext}%
Note an important difference between Section \ref{PaperProof} and our formalization. Although 
a \isa{weighted{\isacharunderscore}pair{\isacharunderscore}graph} is symmetric, the edge set contains both ``directions" of an edge, 
i.e., $(v_1,v_2)$ and $(v_2,v_1)$ are both in \isa{parcs\ G}. Thus, the maximum number of edges (in the 
case that the graph is complete) is $n\cdot(n-1)$ and not $\frac{n\cdot(n-1)}{2}$. Another consequence is that
the number $q$ of edges is always even.%
\end{isamarkuptext}\isamarkuptrue%
\isacommand{lemma}\isamarkupfalse%
\ {\isacharparenleft}\isakeyword{in}\ weighted{\isacharunderscore}pair{\isacharunderscore}graph{\isacharparenright}\ max{\isacharunderscore}arcs{\isacharcolon}\ \isanewline
\ \ \isakeyword{shows}\ {\isachardoublequoteopen}card\ {\isacharparenleft}parcs\ G{\isacharparenright}\ {\isasymle}\ n{\isacharasterisk}{\isacharparenleft}n{\isacharminus}{\isadigit{1}}{\isacharparenright}{\isachardoublequoteclose}%
\isadelimproof
%
\endisadelimproof
%
\isatagproof
%
\endisatagproof
{\isafoldproof}%
%
\isadelimproof
%
\endisadelimproof
%
\isadelimproof
%
\endisadelimproof
%
\isatagproof
%
\endisatagproof
{\isafoldproof}%
%
\isadelimproof
%
\endisadelimproof
%
\isadelimproof
%
\endisadelimproof
%
\isatagproof
%
\endisatagproof
{\isafoldproof}%
%
\isadelimproof
%
\endisadelimproof
%
\isadelimproof
%
\endisadelimproof
%
\isatagproof
%
\endisatagproof
{\isafoldproof}%
%
\isadelimproof
\isanewline
%
\endisadelimproof
\isanewline
\isacommand{lemma}\isamarkupfalse%
\ {\isacharparenleft}\isakeyword{in}\ weighted{\isacharunderscore}pair{\isacharunderscore}graph{\isacharparenright}\ even{\isacharunderscore}arcs{\isacharcolon}\ \isanewline
\isakeyword{shows}\ {\isachardoublequoteopen}even\ q{\isachardoublequoteclose}%
\isadelimproof
%
\endisadelimproof
%
\isatagproof
%
\endisatagproof
{\isafoldproof}%
%
\isadelimproof
%
\endisadelimproof
%
\begin{isamarkuptext}%
The below sublocale \isa{distinct{\isacharunderscore}weighted{\isacharunderscore}pair{\isacharunderscore}graph} refines
\isa{weighted{\isacharunderscore}pair{\isacharunderscore}graph}. The condition 
\isa{zero} fixes the meaning of 0.
The weight function is defined on the set of all vertices but since self loops are not allowed; 
we use 0 as a special value to indicate the unavailability of the edge. 
The second condition \isa{distinct} enforces that no two edges can have the same weight. There
are some exceptions however captured in the statement \isa{{\isacharparenleft}v\isactrlsub {\isadigit{1}}\ {\isacharequal}\ u\isactrlsub {\isadigit{2}}\ {\isasymand}\ v\isactrlsub {\isadigit{2}}\ {\isacharequal}\ u\isactrlsub {\isadigit{1}}{\isacharparenright}\ {\isasymor}\ {\isacharparenleft}v\isactrlsub {\isadigit{1}}\ {\isacharequal}\ u\isactrlsub {\isadigit{1}}\ {\isasymand}\ v\isactrlsub {\isadigit{2}}\ {\isacharequal}\ u\isactrlsub {\isadigit{2}}{\isacharparenright}}.
Firstly, $(v_1,v_2)$ should have the same weight as $(v_2,v_1)$. Secondly, $w(v_1,v_2)$ has the
same value as $w(v_1,v_2)$. Note that both edges being self loops resulting in them both having 
weight 0 is prohibited by condition \isa{zero}.
Our decision to separate these two conditions from the ones in \isa{weighted{\isacharunderscore}pair{\isacharunderscore}graph}
instead of making one locale of its own is two-fold: On the one hand, there are scenarios where 
distinctiveness is not wished for. On the other hand, 0 might not be available as a special value.%
\end{isamarkuptext}\isamarkuptrue%
\isacommand{locale}\isamarkupfalse%
\ distinct{\isacharunderscore}weighted{\isacharunderscore}pair{\isacharunderscore}graph\ {\isacharequal}\ weighted{\isacharunderscore}pair{\isacharunderscore}graph\ {\isacharplus}\ \isanewline
\ \ \isakeyword{assumes}\ zero{\isacharcolon}\ {\isachardoublequoteopen}{\isasymforall}\ v\isactrlsub {\isadigit{1}}\ v\isactrlsub {\isadigit{2}}{\isachardot}\ {\isacharparenleft}v\isactrlsub {\isadigit{1}}{\isacharcomma}v\isactrlsub {\isadigit{2}}{\isacharparenright}\ {\isasymnotin}\ parcs\ G\ {\isasymlongleftrightarrow}\ w\ {\isacharparenleft}v\isactrlsub {\isadigit{1}}{\isacharcomma}v\isactrlsub {\isadigit{2}}{\isacharparenright}\ {\isacharequal}\ {\isadigit{0}}{\isachardoublequoteclose}\isanewline
\ \ \ \ \ \ \isakeyword{and}\ distinct{\isacharcolon}\ {\isachardoublequoteopen}{\isasymforall}\ {\isacharparenleft}v\isactrlsub {\isadigit{1}}{\isacharcomma}v\isactrlsub {\isadigit{2}}{\isacharparenright}\ {\isasymin}\ parcs\ G{\isachardot}\ {\isasymforall}\ {\isacharparenleft}u{\isadigit{1}}{\isacharcomma}u{\isadigit{2}}{\isacharparenright}\ {\isasymin}\ parcs\ G{\isachardot}\ \isanewline
\ \ \ \ \ \ {\isacharparenleft}{\isacharparenleft}v\isactrlsub {\isadigit{1}}\ {\isacharequal}\ u{\isadigit{2}}\ {\isasymand}\ v\isactrlsub {\isadigit{2}}\ {\isacharequal}\ u{\isadigit{1}}{\isacharparenright}\ {\isasymor}\ {\isacharparenleft}v\isactrlsub {\isadigit{1}}\ {\isacharequal}\ u{\isadigit{1}}\ {\isasymand}\ v\isactrlsub {\isadigit{2}}\ {\isacharequal}\ u{\isadigit{2}}{\isacharparenright}{\isacharparenright}\ {\isasymlongleftrightarrow}\ w\ {\isacharparenleft}v\isactrlsub {\isadigit{1}}{\isacharcomma}v\isactrlsub {\isadigit{2}}{\isacharparenright}\ {\isacharequal}\ w\ {\isacharparenleft}u{\isadigit{1}}{\isacharcomma}u{\isadigit{2}}{\isacharparenright}{\isachardoublequoteclose}\ \isanewline
%
\isadelimproof
%
\endisadelimproof
%
\isatagproof
%
\endisatagproof
{\isafoldproof}%
%
\isadelimproof
%
\endisadelimproof
%
\isadelimproof
%
\endisadelimproof
%
\isatagproof
%
\endisatagproof
{\isafoldproof}%
%
\isadelimproof
%
\endisadelimproof
%
\isadelimproof
%
\endisadelimproof
%
\isatagproof
%
\endisatagproof
{\isafoldproof}%
%
\isadelimproof
%
\endisadelimproof
%
\isadelimproof
%
\endisadelimproof
%
\isatagproof
%
\endisatagproof
{\isafoldproof}%
%
\isadelimproof
%
\endisadelimproof
%
\isadelimproof
%
\endisadelimproof
%
\isatagproof
%
\endisatagproof
{\isafoldproof}%
%
\isadelimproof
%
\endisadelimproof
%
\isadelimproof
%
\endisadelimproof
%
\isatagproof
%
\endisatagproof
{\isafoldproof}%
%
\isadelimproof
%
\endisadelimproof
%
\isadelimproof
%
\endisadelimproof
%
\isatagproof
%
\endisatagproof
{\isafoldproof}%
%
\isadelimproof
%
\endisadelimproof
%
\isadelimproof
%
\endisadelimproof
%
\isatagproof
%
\endisatagproof
{\isafoldproof}%
%
\isadelimproof
%
\endisadelimproof
%
\begin{isamarkuptext}%
One important step in our formalization is to show that the weight function is surjective. However, having two 
elements of the domain (edges) being mapped to the same element of the codomain (weight) makes 
the proof complicated. We therefore first prove that the weight function is surjective on a restricted
set of edges. Here we use the fact that there is a linear order on vertices by only considering edges
were the first endpoint is bigger than the second. 

Then, the surjectivity of $w$ is relatively simple to show. Note that we could also have assumed surjectivity in 
\isa{distinct{\isacharunderscore}weighted{\isacharunderscore}pair{\isacharunderscore}graph} and shown that distinctiveness follows from it. However,
distinctiveness is the more natural assumption that is more likely to appear in any application
of ordered trails.%
\end{isamarkuptext}\isamarkuptrue%
\isacommand{lemma}\isamarkupfalse%
{\isacharparenleft}\isakeyword{in}\ distinct{\isacharunderscore}weighted{\isacharunderscore}pair{\isacharunderscore}graph{\isacharparenright}\ restricted{\isacharunderscore}weight{\isacharunderscore}fun{\isacharunderscore}surjective{\isacharcolon}\ \ \isanewline
\ \ \isakeyword{shows}\ {\isachardoublequoteopen}{\isacharparenleft}{\isasymforall}k\ {\isasymin}\ W{\isachardot}\ {\isasymexists}{\isacharparenleft}v\isactrlsub {\isadigit{1}}{\isacharcomma}v\isactrlsub {\isadigit{2}}{\isacharparenright}\ {\isasymin}\ {\isacharparenleft}parcs\ G{\isacharparenright}{\isachardot}\ w\ {\isacharparenleft}v\isactrlsub {\isadigit{1}}{\isacharcomma}v\isactrlsub {\isadigit{2}}{\isacharparenright}\ {\isacharequal}\ k{\isacharparenright}{\isachardoublequoteclose}%
\isadelimproof
%
\endisadelimproof
%
\isatagproof
%
\endisatagproof
{\isafoldproof}%
%
\isadelimproof
%
\endisadelimproof
\isanewline
\isanewline
\isacommand{lemma}\isamarkupfalse%
{\isacharparenleft}\isakeyword{in}\ distinct{\isacharunderscore}weighted{\isacharunderscore}pair{\isacharunderscore}graph{\isacharparenright}\ weight{\isacharunderscore}fun{\isacharunderscore}surjective{\isacharcolon}\isanewline
\ \ \isakeyword{shows}\ {\isachardoublequoteopen}{\isacharparenleft}{\isasymforall}k\ {\isasymin}\ W{\isachardot}\ {\isasymexists}{\isacharparenleft}v\isactrlsub {\isadigit{1}}{\isacharcomma}v\isactrlsub {\isadigit{2}}{\isacharparenright}\ {\isasymin}\ {\isacharparenleft}parcs\ G{\isacharparenright}{\isachardot}\ w\ {\isacharparenleft}v\isactrlsub {\isadigit{1}}{\isacharcomma}v\isactrlsub {\isadigit{2}}{\isacharparenright}\ {\isacharequal}\ k{\isacharparenright}{\isachardoublequoteclose}%
\isadelimproof
%
\endisadelimproof
%
\isatagproof
%
\endisatagproof
{\isafoldproof}%
%
\isadelimproof
%
\endisadelimproof
%
\isadelimproof
%
\endisadelimproof
%
\isatagproof
%
\endisatagproof
{\isafoldproof}%
%
\isadelimproof
%
\endisadelimproof
%
\isadelimproof
%
\endisadelimproof
%
\isatagproof
%
\endisatagproof
{\isafoldproof}%
%
\isadelimproof
%
\endisadelimproof
%
\isadelimproof
%
\endisadelimproof
%
\isatagproof
%
\endisatagproof
{\isafoldproof}%
%
\isadelimproof
%
\endisadelimproof
%
\isadelimproof
%
\endisadelimproof
%
\isatagproof
%
\endisatagproof
{\isafoldproof}%
%
\isadelimproof
%
\endisadelimproof
%
\isadelimproof
%
\endisadelimproof
%
\isatagproof
%
\endisatagproof
{\isafoldproof}%
%
\isadelimproof
%
\endisadelimproof
%
\isadelimdocument
%
\endisadelimdocument
%
\isatagdocument
%
\isamarkupsubsection{Computing a Longest Ordered Trail%
}
\isamarkuptrue%
%
\endisatagdocument
{\isafolddocument}%
%
\isadelimdocument
%
\endisadelimdocument
%
\begin{isamarkuptext}%
\label{compAlgo}We next formally verify Algorithm \ref{algo} and compute longest ordered trails. To this end, 
we introduce the function \isa{findEdge} to find an edge in a list of edges by its weight.%
\end{isamarkuptext}\isamarkuptrue%
\isacommand{fun}\isamarkupfalse%
\ findEdge\ {\isacharcolon}{\isacharcolon}\ {\isachardoublequoteopen}{\isacharparenleft}{\isacharprime}a{\isasymtimes}{\isacharprime}a{\isacharparenright}\ weight{\isacharunderscore}fun\ {\isasymRightarrow}\ {\isacharparenleft}{\isacharprime}a{\isasymtimes}{\isacharprime}a{\isacharparenright}\ list\ {\isasymRightarrow}\ real\ {\isasymRightarrow}\ {\isacharparenleft}{\isacharprime}a{\isasymtimes}{\isacharprime}a{\isacharparenright}{\isachardoublequoteclose}\ \isakeyword{where}\isanewline
{\isachardoublequoteopen}findEdge\ f\ {\isacharbrackleft}{\isacharbrackright}\ k\ {\isacharequal}\ undefined{\isachardoublequoteclose}\ {\isacharbar}\isanewline
{\isachardoublequoteopen}findEdge\ f\ {\isacharparenleft}e{\isacharhash}es{\isacharparenright}\ k\ {\isacharequal}\ {\isacharparenleft}if\ f\ e\ {\isacharequal}\ k\ then\ e\ else\ findEdge\ f\ es\ k{\isacharparenright}{\isachardoublequoteclose}%
\isadelimproof
%
\endisadelimproof
%
\isatagproof
%
\endisatagproof
{\isafoldproof}%
%
\isadelimproof
%
\endisadelimproof
%
\begin{isamarkuptext}%
Function \isa{findEdge} will correctly return the edge whose weight is $k$. We do not care in which order the endpoints
are found, i.e. whether $(v_1,v_2)$ or $(v_2,v_1)$ is returned.%
\end{isamarkuptext}\isamarkuptrue%
\isacommand{lemma}\isamarkupfalse%
{\isacharparenleft}\isakeyword{in}\ distinct{\isacharunderscore}weighted{\isacharunderscore}pair{\isacharunderscore}graph{\isacharparenright}\ findEdge{\isacharunderscore}success{\isacharcolon}\isanewline
\ \ \isakeyword{assumes}\ {\isachardoublequoteopen}k\ {\isasymin}\ W{\isachardoublequoteclose}\ \isakeyword{and}\ {\isachardoublequoteopen}w\ {\isacharparenleft}v\isactrlsub {\isadigit{1}}{\isacharcomma}v\isactrlsub {\isadigit{2}}{\isacharparenright}\ {\isacharequal}\ k{\isachardoublequoteclose}\ \isakeyword{and}\ {\isachardoublequoteopen}{\isacharparenleft}parcs\ G{\isacharparenright}\ {\isasymnoteq}\ {\isacharbraceleft}{\isacharbraceright}{\isachardoublequoteclose}\ \isanewline
\ \ \isakeyword{shows}\ {\isachardoublequoteopen}{\isacharparenleft}findEdge\ w\ {\isacharparenleft}set{\isacharunderscore}to{\isacharunderscore}list\ {\isacharparenleft}parcs\ G{\isacharparenright}{\isacharparenright}\ k{\isacharparenright}\ {\isacharequal}\ {\isacharparenleft}v\isactrlsub {\isadigit{1}}{\isacharcomma}v\isactrlsub {\isadigit{2}}{\isacharparenright}\ \isanewline
\ \ \ \ \ \ \ \ {\isasymor}\ {\isacharparenleft}findEdge\ w\ {\isacharparenleft}set{\isacharunderscore}to{\isacharunderscore}list\ {\isacharparenleft}parcs\ G{\isacharparenright}{\isacharparenright}\ k{\isacharparenright}\ {\isacharequal}\ {\isacharparenleft}v\isactrlsub {\isadigit{2}}{\isacharcomma}v\isactrlsub {\isadigit{1}}{\isacharparenright}{\isachardoublequoteclose}%
\isadelimproof
%
\endisadelimproof
%
\isatagproof
%
\endisatagproof
{\isafoldproof}%
%
\isadelimproof
%
\endisadelimproof
%
\isadelimproof
%
\endisadelimproof
%
\isatagproof
%
\endisatagproof
{\isafoldproof}%
%
\isadelimproof
%
\endisadelimproof
%
\isadelimproof
%
\endisadelimproof
%
\isatagproof
%
\endisatagproof
{\isafoldproof}%
%
\isadelimproof
%
\endisadelimproof
%
\begin{isamarkuptext}%
We translate the notion of a labelling function $L^i(v)$ (see Definition \ref{Labelling}) into Isabelle.
Function \isa{getL\ G\ w}, in short for get label, returns the length of the longest strictly decreasing
path starting at vertex $v$. In contrast to Definition \ref{Labelling} subgraphs are treated here implicitly. Intuitively,
this can be seen as adding edges to an empty graph in order of their weight.%
\end{isamarkuptext}\isamarkuptrue%
\isacommand{fun}\isamarkupfalse%
\ getL\ {\isacharcolon}{\isacharcolon}\ {\isachardoublequoteopen}{\isacharparenleft}{\isacharprime}a{\isacharcolon}{\isacharcolon}linorder{\isacharparenright}\ pair{\isacharunderscore}pre{\isacharunderscore}digraph\ {\isasymRightarrow}\ {\isacharparenleft}{\isacharprime}a{\isasymtimes}{\isacharprime}a{\isacharparenright}\ weight{\isacharunderscore}fun\ {\isasymRightarrow}\ nat\ {\isasymRightarrow}\ {\isacharprime}a\ {\isasymRightarrow}\ nat{\isachardoublequoteclose}\ \isakeyword{where}\isanewline
{\isachardoublequoteopen}getL\ g\ w\ {\isadigit{0}}\ v\ {\isacharequal}\ {\isadigit{0}}{\isachardoublequoteclose}\ {\isacharbar}\ \ \isanewline
{\isachardoublequoteopen}getL\ g\ w\ {\isacharparenleft}Suc\ i{\isacharparenright}\ v\ {\isacharequal}\ {\isacharparenleft}let\ {\isacharparenleft}v\isactrlsub {\isadigit{1}}{\isacharcomma}v\isactrlsub {\isadigit{2}}{\isacharparenright}\ {\isacharequal}\ {\isacharparenleft}findEdge\ w\ {\isacharparenleft}set{\isacharunderscore}to{\isacharunderscore}list\ {\isacharparenleft}arcs\ g{\isacharparenright}{\isacharparenright}\ {\isacharparenleft}Suc\ i{\isacharparenright}{\isacharparenright}\ in\ \isanewline
\ \ \ \ \ \ \ \ \ \ \ \ \ \ \ \ \ \ \ {\isacharparenleft}if\ v\ {\isacharequal}\ v\isactrlsub {\isadigit{1}}\ then\ max\ {\isacharparenleft}{\isacharparenleft}getL\ g\ w\ i\ v\isactrlsub {\isadigit{2}}{\isacharparenright}{\isacharplus}{\isadigit{1}}{\isacharparenright}\ {\isacharparenleft}getL\ g\ w\ i\ v{\isacharparenright}\ else\ \isanewline
\ \ \ \ \ \ \ \ \ \ \ \ \ \ \ \ \ \ \ {\isacharparenleft}if\ v\ {\isacharequal}\ v\isactrlsub {\isadigit{2}}\ then\ max\ {\isacharparenleft}{\isacharparenleft}getL\ g\ w\ i\ v\isactrlsub {\isadigit{1}}{\isacharparenright}{\isacharplus}{\isadigit{1}}{\isacharparenright}\ {\isacharparenleft}getL\ g\ w\ i\ v{\isacharparenright}\ else\ \isanewline
\ \ \ \ \ \ \ \ \ \ \ \ \ \ \ \ \ \ \ \ getL\ g\ w\ i\ v{\isacharparenright}{\isacharparenright}{\isacharparenright}{\isachardoublequoteclose}%
\begin{isamarkuptext}%
To add all edges to the graph, set $i=|E|$. Recall that \isa{card\ {\isacharparenleft}parcs\ g{\isacharparenright}} $= 2*|E|$, 
as every edge appears twice. 
Then, iterate over all vertices and give back the
maximum length which is found by using \isa{getL\ G\ w}. Since \isa{getL\ G\ w} can also be used to get a longest 
strictly increasing trail ending at vertex $v$ the algorithm is not restricted to strictly decreasing trails.%
\end{isamarkuptext}\isamarkuptrue%
\isacommand{definition}\isamarkupfalse%
\ getLongestTrail\ {\isacharcolon}{\isacharcolon}\ \isanewline
{\isachardoublequoteopen}{\isacharparenleft}{\isacharprime}a{\isacharcolon}{\isacharcolon}linorder{\isacharparenright}\ pair{\isacharunderscore}pre{\isacharunderscore}digraph\ {\isasymRightarrow}\ {\isacharparenleft}{\isacharprime}a{\isasymtimes}{\isacharprime}a{\isacharparenright}\ weight{\isacharunderscore}fun\ {\isasymRightarrow}\ nat{\isachardoublequoteclose}\ \isakeyword{where}\isanewline
{\isachardoublequoteopen}getLongestTrail\ g\ w\ {\isacharequal}\ \isanewline
Max\ {\isacharparenleft}set\ {\isacharbrackleft}{\isacharparenleft}getL\ g\ w\ {\isacharparenleft}card\ {\isacharparenleft}parcs\ g{\isacharparenright}\ div\ {\isadigit{2}}{\isacharparenright}\ v{\isacharparenright}\ {\isachardot}\ v\ {\isacharless}{\isacharminus}\ sorted{\isacharunderscore}list{\isacharunderscore}of{\isacharunderscore}set\ {\isacharparenleft}pverts\ g{\isacharparenright}{\isacharbrackright}{\isacharparenright}{\isachardoublequoteclose}%
\begin{isamarkuptext}%
Exporting the algorithm into Haskell code results in a fully verified program to find a longest
strictly decreasing or strictly increasing trail.%
\end{isamarkuptext}\isamarkuptrue%
\isacommand{export{\isacharunderscore}code}\isamarkupfalse%
\ getLongestTrail\ \isakeyword{in}\ Haskell\ \isakeyword{module{\isacharunderscore}name}\ LongestTrail%
\isadelimproof
%
\endisadelimproof
%
\isatagproof
%
\endisatagproof
{\isafoldproof}%
%
\isadelimproof
%
\endisadelimproof
%
\isadelimproof
%
\endisadelimproof
%
\isatagproof
%
\endisatagproof
{\isafoldproof}%
%
\isadelimproof
%
\endisadelimproof
%
\isadelimproof
%
\endisadelimproof
%
\isatagproof
%
\endisatagproof
{\isafoldproof}%
%
\isadelimproof
%
\endisadelimproof
%
\isadelimproof
%
\endisadelimproof
%
\isatagproof
%
\endisatagproof
{\isafoldproof}%
%
\isadelimproof
%
\endisadelimproof
%
\begin{isamarkuptext}%
Using an induction proof and extensive case distinction, the correctness of Algorithm \ref{algo} 
is then shown in our formalization, by proving the following theorem:%
\end{isamarkuptext}\isamarkuptrue%
\isacommand{theorem}\isamarkupfalse%
{\isacharparenleft}\isakeyword{in}\ distinct{\isacharunderscore}weighted{\isacharunderscore}pair{\isacharunderscore}graph{\isacharparenright}\ \ correctness{\isacharcolon}\isanewline
\ \ \isakeyword{assumes}\ {\isachardoublequoteopen}{\isasymexists}\ v\ {\isasymin}\ {\isacharparenleft}pverts\ G{\isacharparenright}{\isachardot}\ getL\ G\ w\ {\isacharparenleft}q\ div\ {\isadigit{2}}{\isacharparenright}\ v\ {\isacharequal}\ k{\isachardoublequoteclose}\isanewline
\ \ \isakeyword{shows}\ {\isachardoublequoteopen}{\isasymexists}\ xs{\isachardot}\ decTrail\ G\ w\ xs\ {\isasymand}\ length\ xs\ {\isacharequal}\ k{\isachardoublequoteclose}%
\isadelimproof
%
\endisadelimproof
%
\isatagproof
%
\endisatagproof
{\isafoldproof}%
%
\isadelimproof
%
\endisadelimproof
%
\isadelimdocument
%
\endisadelimdocument
%
\isatagdocument
%
\isamarkupsubsection{Minimum Length of Ordered Trails%
}
\isamarkuptrue%
%
\endisatagdocument
{\isafolddocument}%
%
\isadelimdocument
%
\endisadelimdocument
%
\isadelimproof
%
\endisadelimproof
%
\isatagproof
%
\endisatagproof
{\isafoldproof}%
%
\isadelimproof
%
\endisadelimproof
%
\begin{isamarkuptext}%
\label{minLength}
The algorithm introduced in Section \ref{compAlgo} is already useful on its own. Additionally, it can be
used to verify the lower bound on the minimum length of a strictly decreasing trail $P_d(w,G) \ge 2 \cdot \lfloor \frac{q}{n} \rfloor$.

To this end, Lemma 1 from Section \ref{PaperProof} is translated into Isabelle as the lemma
\isa{minimal{\isacharunderscore}increase{\isacharunderscore}one{\isacharunderscore}step}. The proof is 
similar to its counterpart, also using a case distinction. Lemma 2 is subsequently proved, here
named \isa{minimal{\isacharunderscore}increase{\isacharunderscore}total}.%
\end{isamarkuptext}\isamarkuptrue%
\isacommand{lemma}\isamarkupfalse%
{\isacharparenleft}\isakeyword{in}\ distinct{\isacharunderscore}weighted{\isacharunderscore}pair{\isacharunderscore}graph{\isacharparenright}\ minimal{\isacharunderscore}increase{\isacharunderscore}one{\isacharunderscore}step{\isacharcolon}\isanewline
\ \ \isakeyword{assumes}\ {\isachardoublequoteopen}k\ {\isacharplus}\ {\isadigit{1}}\ {\isasymin}\ W{\isachardoublequoteclose}\isanewline
\ \ \isakeyword{shows}\ \isanewline
\ \ \ \ {\isachardoublequoteopen}{\isacharparenleft}{\isasymSum}\ v\ {\isasymin}\ pverts\ G{\isachardot}\ getL\ G\ w\ {\isacharparenleft}k{\isacharplus}{\isadigit{1}}{\isacharparenright}\ v{\isacharparenright}\ {\isasymge}\ {\isacharparenleft}{\isasymSum}\ v\ {\isasymin}\ pverts\ G{\isachardot}\ getL\ G\ w\ k\ v{\isacharparenright}\ {\isacharplus}\ {\isadigit{2}}{\isachardoublequoteclose}%
\isadelimproof
%
\endisadelimproof
%
\isatagproof
%
\endisatagproof
{\isafoldproof}%
%
\isadelimproof
%
\endisadelimproof
%
\isadelimproof
%
\endisadelimproof
%
\isatagproof
%
\endisatagproof
{\isafoldproof}%
%
\isadelimproof
\isanewline
%
\endisadelimproof
\isanewline
\isacommand{lemma}\isamarkupfalse%
{\isacharparenleft}\isakeyword{in}\ distinct{\isacharunderscore}weighted{\isacharunderscore}pair{\isacharunderscore}graph{\isacharparenright}\ minimal{\isacharunderscore}increase{\isacharunderscore}total{\isacharcolon}\isanewline
\ \ \isakeyword{shows}\ {\isachardoublequoteopen}{\isacharparenleft}{\isasymSum}\ v\ {\isasymin}\ pverts\ G{\isachardot}\ getL\ G\ w\ {\isacharparenleft}q\ div\ {\isadigit{2}}{\isacharparenright}\ v{\isacharparenright}\ {\isasymge}\ q{\isachardoublequoteclose}%
\isadelimproof
%
\endisadelimproof
%
\isatagproof
%
\endisatagproof
{\isafoldproof}%
%
\isadelimproof
%
\endisadelimproof
%
\isadelimproof
%
\endisadelimproof
%
\isatagproof
%
\endisatagproof
{\isafoldproof}%
%
\isadelimproof
%
\endisadelimproof
%
\begin{isamarkuptext}%
From \isa{minimal{\isacharunderscore}increase{\isacharunderscore}total} we have that that the sum of all labels after $q$ div $2$ steps is 
greater than $q$. Now assume that all labels are smaller than $q$ div $n$. Because we have $n$ vertices, this
leads to a contradiction, which proves \isa{algo{\isacharunderscore}result{\isacharunderscore}min}.%
\end{isamarkuptext}\isamarkuptrue%
\isacommand{lemma}\isamarkupfalse%
{\isacharparenleft}\isakeyword{in}\ distinct{\isacharunderscore}weighted{\isacharunderscore}pair{\isacharunderscore}graph{\isacharparenright}\ algo{\isacharunderscore}result{\isacharunderscore}min{\isacharcolon}\ \isanewline
\ \ \isakeyword{shows}\ {\isachardoublequoteopen}{\isacharparenleft}{\isasymexists}\ v\ {\isasymin}\ pverts\ G{\isachardot}\ getL\ G\ w\ {\isacharparenleft}q\ div\ {\isadigit{2}}{\isacharparenright}\ v\ {\isasymge}\ q\ div\ n{\isacharparenright}{\isachardoublequoteclose}%
\isadelimproof
%
\endisadelimproof
%
\isatagproof
%
\endisatagproof
{\isafoldproof}%
%
\isadelimproof
%
\endisadelimproof
%
\begin{isamarkuptext}%
Finally, using lemma \isa{algo{\isacharunderscore}result{\isacharunderscore}min} together with the \isa{correctness} theorem 
of section \ref{compAlgo}, we prove the lower bound of $2\cdot\lfloor \frac{q}{n} \rfloor$ over the length 
of a longest strictly decreasing trail. This general approach could also be used to extend our
formalization and prove existence of other trails. For example, assume that some restrictions on the graph 
give raise to the existence of a trail of length $m \ge 2\cdot\lfloor \frac{q}{n} \rfloor$. Then, it is
only necessary to show that our algorithm can find this trail.%
\end{isamarkuptext}\isamarkuptrue%
\isacommand{theorem}\isamarkupfalse%
{\isacharparenleft}\isakeyword{in}\ distinct{\isacharunderscore}weighted{\isacharunderscore}pair{\isacharunderscore}graph{\isacharparenright}\ dec{\isacharunderscore}trail{\isacharunderscore}exists{\isacharcolon}\ \isanewline
\ \ \isakeyword{shows}\ {\isachardoublequoteopen}{\isasymexists}\ es{\isachardot}\ decTrail\ G\ w\ es\ {\isasymand}\ length\ es\ {\isacharequal}\ q\ div\ n{\isachardoublequoteclose}%
\isadelimproof
%
\endisadelimproof
%
\isatagproof
%
\endisatagproof
{\isafoldproof}%
%
\isadelimproof
%
\endisadelimproof
\isanewline
\isanewline
\isacommand{theorem}\isamarkupfalse%
{\isacharparenleft}\isakeyword{in}\ distinct{\isacharunderscore}weighted{\isacharunderscore}pair{\isacharunderscore}graph{\isacharparenright}\ inc{\isacharunderscore}trail{\isacharunderscore}exists{\isacharcolon}\ \isanewline
\ \ \isakeyword{shows}\ {\isachardoublequoteopen}{\isasymexists}\ es{\isachardot}\ incTrail\ G\ w\ es\ {\isasymand}\ length\ es\ {\isacharequal}\ q\ div\ n{\isachardoublequoteclose}\isanewline
\ \ %
\isadelimproof
%
\endisadelimproof
%
\isatagproof
%
\endisatagproof
{\isafoldproof}%
%
\isadelimproof
%
\endisadelimproof
%
\begin{isamarkuptext}%
Corollary 1 is translated into \isa{dec{\isacharunderscore}trail{\isacharunderscore}exists{\isacharunderscore}complete}. The proof first argues
that the number of edges is $n\cdot(n-1)$ by restricting its domain as done already in Section \ref{localeSurjective}.%
\end{isamarkuptext}\isamarkuptrue%
\isacommand{lemma}\isamarkupfalse%
{\isacharparenleft}\isakeyword{in}\ distinct{\isacharunderscore}weighted{\isacharunderscore}pair{\isacharunderscore}graph{\isacharparenright}\ dec{\isacharunderscore}trail{\isacharunderscore}exists{\isacharunderscore}complete{\isacharcolon}\ \isanewline
\ \ \isakeyword{assumes}\ {\isachardoublequoteopen}complete{\isacharunderscore}digraph\ n\ G{\isachardoublequoteclose}\ \isanewline
\ \ \isakeyword{shows}\ {\isachardoublequoteopen}{\isacharparenleft}{\isasymexists}\ es{\isachardot}\ decTrail\ G\ w\ es\ {\isasymand}\ length\ es\ {\isacharequal}\ n{\isacharminus}{\isadigit{1}}{\isacharparenright}{\isachardoublequoteclose}%
\isadelimproof
%
\endisadelimproof
%
\isatagproof
%
\endisatagproof
{\isafoldproof}%
%
\isadelimproof
%
\endisadelimproof
%
\isadelimproof
%
\endisadelimproof
%
\isatagproof
%
\endisatagproof
{\isafoldproof}%
%
\isadelimproof
%
\endisadelimproof
%
\isadelimdocument
%
\endisadelimdocument
%
\isatagdocument
%
\isamarkupsubsection{Example Graph $K_4$%
}
\isamarkuptrue%
%
\endisatagdocument
{\isafolddocument}%
%
\isadelimdocument
%
\endisadelimdocument
%
\begin{isamarkuptext}%
We return to the example graph from Figure \ref{exampleGraph} and show that our results from 
Sections \ref{trails}-\ref{minLength} can be used to prove existence of trails of length $k$, in particular
$k = 3$ in $K_4$. Defining the graph and the 
weight function separately, we use natural numbers as vertices.%
\end{isamarkuptext}\isamarkuptrue%
\isacommand{abbreviation}\isamarkupfalse%
\ ExampleGraph{\isacharcolon}{\isacharcolon}\ {\isachardoublequoteopen}nat\ pair{\isacharunderscore}pre{\isacharunderscore}digraph{\isachardoublequoteclose}\ \isakeyword{where}\ \isanewline
{\isachardoublequoteopen}ExampleGraph\ {\isasymequiv}\ {\isacharparenleft}{\isacharbar}\ \isanewline
pverts\ {\isacharequal}\ {\isacharbraceleft}{\isadigit{1}}{\isacharcomma}{\isadigit{2}}{\isacharcomma}{\isadigit{3}}{\isacharcomma}{\isacharparenleft}{\isadigit{4}}{\isacharcolon}{\isacharcolon}nat{\isacharparenright}{\isacharbraceright}{\isacharcomma}\ \isanewline
parcs\ {\isacharequal}\ {\isacharbraceleft}{\isacharparenleft}v\isactrlsub {\isadigit{1}}{\isacharcomma}v\isactrlsub {\isadigit{2}}{\isacharparenright}{\isachardot}\ v\isactrlsub {\isadigit{1}}\ {\isasymin}\ {\isacharbraceleft}{\isadigit{1}}{\isacharcomma}{\isadigit{2}}{\isacharcomma}{\isadigit{3}}{\isacharcomma}{\isacharparenleft}{\isadigit{4}}{\isacharcolon}{\isacharcolon}nat{\isacharparenright}{\isacharbraceright}\ {\isasymand}\ v\isactrlsub {\isadigit{2}}\ {\isasymin}\ {\isacharbraceleft}{\isadigit{1}}{\isacharcomma}{\isadigit{2}}{\isacharcomma}{\isadigit{3}}{\isacharcomma}{\isacharparenleft}{\isadigit{4}}{\isacharcolon}{\isacharcolon}nat{\isacharparenright}{\isacharbraceright}\ {\isasymand}\ v\isactrlsub {\isadigit{1}}\ {\isasymnoteq}\ v\isactrlsub {\isadigit{2}}{\isacharbraceright}\ \isanewline
{\isacharbar}{\isacharparenright}{\isachardoublequoteclose}\ \isanewline
\isanewline
\isacommand{abbreviation}\isamarkupfalse%
\ ExampleGraphWeightFunction\ {\isacharcolon}{\isacharcolon}\ {\isachardoublequoteopen}{\isacharparenleft}nat{\isasymtimes}nat{\isacharparenright}\ weight{\isacharunderscore}fun{\isachardoublequoteclose}\ \isakeyword{where}\ \isanewline
{\isachardoublequoteopen}ExampleGraphWeightFunction\ {\isasymequiv}\ {\isacharparenleft}{\isasymlambda}{\isacharparenleft}v\isactrlsub {\isadigit{1}}{\isacharcomma}v\isactrlsub {\isadigit{2}}{\isacharparenright}{\isachardot}\ \isanewline
\ \ \ \ \ \ \ \ \ \ \ \ \ \ \ \ \ \ \ \ \ \ \ \ \ \ \ \ \ \ \ if\ {\isacharparenleft}v\isactrlsub {\isadigit{1}}\ {\isacharequal}\ {\isadigit{1}}\ {\isasymand}\ v\isactrlsub {\isadigit{2}}\ {\isacharequal}\ {\isadigit{2}}{\isacharparenright}\ {\isasymor}\ {\isacharparenleft}v\isactrlsub {\isadigit{1}}\ {\isacharequal}\ {\isadigit{2}}\ {\isasymand}\ v\isactrlsub {\isadigit{2}}\ {\isacharequal}\ {\isadigit{1}}{\isacharparenright}\ then\ {\isadigit{1}}\ else\isanewline
\ \ \ \ \ \ \ \ \ \ \ \ \ \ \ \ \ \ \ \ \ \ \ \ \ \ \ \ \ \ {\isacharparenleft}if\ {\isacharparenleft}v\isactrlsub {\isadigit{1}}\ {\isacharequal}\ {\isadigit{1}}\ {\isasymand}\ v\isactrlsub {\isadigit{2}}\ {\isacharequal}\ {\isadigit{3}}{\isacharparenright}\ {\isasymor}\ {\isacharparenleft}v\isactrlsub {\isadigit{1}}\ {\isacharequal}\ {\isadigit{3}}\ {\isasymand}\ v\isactrlsub {\isadigit{2}}\ {\isacharequal}\ {\isadigit{1}}{\isacharparenright}\ then\ {\isadigit{3}}\ else\isanewline
\ \ \ \ \ \ \ \ \ \ \ \ \ \ \ \ \ \ \ \ \ \ \ \ \ \ \ \ \ \ {\isacharparenleft}if\ {\isacharparenleft}v\isactrlsub {\isadigit{1}}\ {\isacharequal}\ {\isadigit{1}}\ {\isasymand}\ v\isactrlsub {\isadigit{2}}\ {\isacharequal}\ {\isadigit{4}}{\isacharparenright}\ {\isasymor}\ {\isacharparenleft}v\isactrlsub {\isadigit{1}}\ {\isacharequal}\ {\isadigit{4}}\ {\isasymand}\ v\isactrlsub {\isadigit{2}}\ {\isacharequal}\ {\isadigit{1}}{\isacharparenright}\ then\ {\isadigit{6}}\ else\isanewline
\ \ \ \ \ \ \ \ \ \ \ \ \ \ \ \ \ \ \ \ \ \ \ \ \ \ \ \ \ \ {\isacharparenleft}if\ {\isacharparenleft}v\isactrlsub {\isadigit{1}}\ {\isacharequal}\ {\isadigit{2}}\ {\isasymand}\ v\isactrlsub {\isadigit{2}}\ {\isacharequal}\ {\isadigit{3}}{\isacharparenright}\ {\isasymor}\ {\isacharparenleft}v\isactrlsub {\isadigit{1}}\ {\isacharequal}\ {\isadigit{3}}\ {\isasymand}\ v\isactrlsub {\isadigit{2}}\ {\isacharequal}\ {\isadigit{2}}{\isacharparenright}\ then\ {\isadigit{5}}\ else\ \isanewline
\ \ \ \ \ \ \ \ \ \ \ \ \ \ \ \ \ \ \ \ \ \ \ \ \ \ \ \ \ \ {\isacharparenleft}if\ {\isacharparenleft}v\isactrlsub {\isadigit{1}}\ {\isacharequal}\ {\isadigit{2}}\ {\isasymand}\ v\isactrlsub {\isadigit{2}}\ {\isacharequal}\ {\isadigit{4}}{\isacharparenright}\ {\isasymor}\ {\isacharparenleft}v\isactrlsub {\isadigit{1}}\ {\isacharequal}\ {\isadigit{4}}\ {\isasymand}\ v\isactrlsub {\isadigit{2}}\ {\isacharequal}\ {\isadigit{2}}{\isacharparenright}\ then\ {\isadigit{4}}\ else\isanewline
\ \ \ \ \ \ \ \ \ \ \ \ \ \ \ \ \ \ \ \ \ \ \ \ \ \ \ \ \ \ {\isacharparenleft}if\ {\isacharparenleft}v\isactrlsub {\isadigit{1}}\ {\isacharequal}\ {\isadigit{3}}\ {\isasymand}\ v\isactrlsub {\isadigit{2}}\ {\isacharequal}\ {\isadigit{4}}{\isacharparenright}\ {\isasymor}\ {\isacharparenleft}v\isactrlsub {\isadigit{1}}\ {\isacharequal}\ {\isadigit{4}}\ {\isasymand}\ v\isactrlsub {\isadigit{2}}\ {\isacharequal}\ {\isadigit{3}}{\isacharparenright}\ then\ {\isadigit{2}}\ else\ \isanewline
\ \ \ \ \ \ \ \ \ \ \ \ \ \ \ \ \ \ \ \ \ \ \ \ \ \ \ \ \ \ \ {\isadigit{0}}{\isacharparenright}{\isacharparenright}{\isacharparenright}{\isacharparenright}{\isacharparenright}{\isacharparenright}{\isachardoublequoteclose}\ \isanewline
%
\isadelimproof
%
\endisadelimproof
%
\isatagproof
%
\endisatagproof
{\isafoldproof}%
%
\isadelimproof
%
\endisadelimproof
%
\isadelimproof
%
\endisadelimproof
%
\isatagproof
%
\endisatagproof
{\isafoldproof}%
%
\isadelimproof
%
\endisadelimproof
%
\isadelimproof
%
\endisadelimproof
%
\isatagproof
%
\endisatagproof
{\isafoldproof}%
%
\isadelimproof
%
\endisadelimproof
%
\begin{isamarkuptext}%
We show that the graph $K_4$ of Figure \ref{exampleGraph} satisfies the conditions that were
imposed in 
\isa{distinct{\isacharunderscore}weighted{\isacharunderscore}pair{\isacharunderscore}graph} and its parent locale, including for example no self loops 
and distinctiveness. Of course there is still some effort required for this. However, it is necessary
to manually construct trails or list all possible weight distributions. Additionally, instead of 
$q!$ statements there are at most $\frac{3q}{2}$ statements needed.%
\end{isamarkuptext}\isamarkuptrue%
\isacommand{interpretation}\isamarkupfalse%
\ example{\isacharcolon}\ \isanewline
\ \ distinct{\isacharunderscore}weighted{\isacharunderscore}pair{\isacharunderscore}graph\ ExampleGraph\ ExampleGraphWeightFunction%
\isadelimproof
%
\endisadelimproof
%
\isatagproof
%
\endisatagproof
{\isafoldproof}%
%
\isadelimproof
%
\endisadelimproof
%
\begin{isamarkuptext}%
Now it is an easy task to prove that there is a trail of length 3. We only add the fact that
\isa{ExampleGraph} is a \isa{distinct{\isacharunderscore}weighted{\isacharunderscore}pair{\isacharunderscore}graph} and lemma \isa{dec{\isacharunderscore}trail{\isacharunderscore}exists}.%
\end{isamarkuptext}\isamarkuptrue%
\isacommand{lemma}\isamarkupfalse%
\ ExampleGraph{\isacharunderscore}decTrail{\isacharcolon}\isanewline
\ \ {\isachardoublequoteopen}{\isasymexists}\ xs{\isachardot}\ decTrail\ ExampleGraph\ ExampleGraphWeightFunction\ xs\ {\isasymand}\ length\ xs\ {\isacharequal}\ {\isadigit{3}}{\isachardoublequoteclose}%
\isadelimproof
%
\endisadelimproof
%
\isatagproof
%
\endisatagproof
{\isafoldproof}%
%
\isadelimproof
%
\endisadelimproof
%
\isadelimtheory
%
\endisadelimtheory
%
\isatagtheory
%
\endisatagtheory
{\isafoldtheory}%
%
\isadelimtheory
%
\endisadelimtheory
%
\end{isabellebody}%
\endinput
%:%file=~/Dokumente/Masterarbeit/Test/UploadCode4/Ordered_Trail.thy%:%
%:%24=9%:%
%:%36=11%:%
%:%37=12%:%
%:%38=13%:%
%:%39=14%:%
%:%40=15%:%
%:%41=16%:%
%:%44=16%:%
%:%45=17%:%
%:%46=18%:%
%:%47=19%:%
%:%48=20%:%
%:%49=21%:%
%:%50=22%:%
%:%51=23%:%
%:%52=24%:%
%:%58=27%:%
%:%59=28%:%
%:%60=29%:%
%:%61=30%:%
%:%62=31%:%
%:%63=32%:%
%:%64=33%:%
%:%65=34%:%
%:%70=36%:%
%:%71=37%:%
%:%72=38%:%
%:%73=39%:%
%:%74=40%:%
%:%75=41%:%
%:%76=42%:%
%:%85=44%:%
%:%97=45%:%
%:%106=47%:%
%:%118=49%:%
%:%119=50%:%
%:%120=51%:%
%:%121=52%:%
%:%122=53%:%
%:%123=54%:%
%:%124=55%:%
%:%125=56%:%
%:%126=57%:%
%:%127=58%:%
%:%128=59%:%
%:%129=60%:%
%:%130=61%:%
%:%132=63%:%
%:%133=63%:%
%:%134=64%:%
%:%135=65%:%
%:%136=66%:%
%:%137=67%:%
%:%138=68%:%
%:%139=69%:%
%:%140=69%:%
%:%141=70%:%
%:%142=71%:%
%:%143=73%:%
%:%144=74%:%
%:%145=75%:%
%:%146=75%:%
%:%147=76%:%
%:%148=77%:%
%:%149=78%:%
%:%150=79%:%
%:%151=80%:%
%:%152=81%:%
%:%153=81%:%
%:%154=82%:%
%:%156=84%:%
%:%157=85%:%
%:%158=86%:%
%:%160=88%:%
%:%161=88%:%
%:%162=89%:%
%:%163=90%:%
%:%180=113%:%
%:%181=113%:%
%:%182=114%:%
%:%183=115%:%
%:%328=546%:%
%:%329=547%:%
%:%331=549%:%
%:%332=549%:%
%:%333=550%:%
%:%350=599%:%
%:%351=599%:%
%:%352=600%:%
%:%393=759%:%
%:%394=760%:%
%:%395=761%:%
%:%396=762%:%
%:%397=763%:%
%:%399=765%:%
%:%400=765%:%
%:%401=766%:%
%:%402=767%:%
%:%403=768%:%
%:%416=769%:%
%:%417=770%:%
%:%418=770%:%
%:%419=771%:%
%:%420=772%:%
%:%440=788%:%
%:%452=790%:%
%:%453=791%:%
%:%454=792%:%
%:%455=793%:%
%:%456=794%:%
%:%457=795%:%
%:%458=796%:%
%:%459=797%:%
%:%460=798%:%
%:%461=799%:%
%:%463=801%:%
%:%464=801%:%
%:%465=802%:%
%:%466=803%:%
%:%467=804%:%
%:%469=809%:%
%:%471=811%:%
%:%472=811%:%
%:%473=812%:%
%:%474=812%:%
%:%475=813%:%
%:%476=813%:%
%:%478=815%:%
%:%479=816%:%
%:%480=817%:%
%:%481=818%:%
%:%482=819%:%
%:%484=821%:%
%:%485=821%:%
%:%486=822%:%
%:%536=936%:%
%:%539=937%:%
%:%540=938%:%
%:%541=938%:%
%:%542=939%:%
%:%557=950%:%
%:%558=951%:%
%:%559=952%:%
%:%560=953%:%
%:%561=954%:%
%:%562=955%:%
%:%563=956%:%
%:%564=957%:%
%:%565=958%:%
%:%566=959%:%
%:%567=960%:%
%:%568=961%:%
%:%569=962%:%
%:%571=965%:%
%:%572=965%:%
%:%573=966%:%
%:%574=967%:%
%:%575=968%:%
%:%682=1138%:%
%:%683=1139%:%
%:%684=1140%:%
%:%685=1141%:%
%:%686=1142%:%
%:%687=1143%:%
%:%688=1144%:%
%:%689=1145%:%
%:%690=1146%:%
%:%691=1147%:%
%:%693=1149%:%
%:%694=1149%:%
%:%695=1150%:%
%:%708=1200%:%
%:%709=1201%:%
%:%710=1202%:%
%:%711=1202%:%
%:%712=1203%:%
%:%797=1249%:%
%:%809=1251%:%
%:%810=1252%:%
%:%812=1254%:%
%:%813=1254%:%
%:%814=1255%:%
%:%815=1256%:%
%:%830=1308%:%
%:%831=1309%:%
%:%833=1311%:%
%:%834=1311%:%
%:%835=1312%:%
%:%836=1313%:%
%:%878=1371%:%
%:%879=1372%:%
%:%880=1373%:%
%:%881=1374%:%
%:%883=1377%:%
%:%884=1377%:%
%:%885=1378%:%
%:%886=1379%:%
%:%891=1384%:%
%:%892=1385%:%
%:%893=1386%:%
%:%894=1387%:%
%:%895=1388%:%
%:%897=1390%:%
%:%898=1390%:%
%:%899=1391%:%
%:%900=1392%:%
%:%903=1395%:%
%:%904=1396%:%
%:%906=1398%:%
%:%907=1398%:%
%:%961=1558%:%
%:%962=1559%:%
%:%964=1561%:%
%:%965=1561%:%
%:%966=1562%:%
%:%967=1563%:%
%:%987=1573%:%
%:%1012=1636%:%
%:%1013=1637%:%
%:%1014=1638%:%
%:%1015=1639%:%
%:%1016=1640%:%
%:%1017=1641%:%
%:%1018=1642%:%
%:%1019=1643%:%
%:%1021=1645%:%
%:%1022=1645%:%
%:%1023=1646%:%
%:%1024=1647%:%
%:%1025=1648%:%
%:%1049=1709%:%
%:%1052=1710%:%
%:%1053=1711%:%
%:%1054=1711%:%
%:%1055=1712%:%
%:%1083=1743%:%
%:%1084=1744%:%
%:%1085=1745%:%
%:%1087=1747%:%
%:%1088=1747%:%
%:%1089=1748%:%
%:%1104=1760%:%
%:%1105=1761%:%
%:%1106=1762%:%
%:%1107=1763%:%
%:%1108=1764%:%
%:%1109=1765%:%
%:%1111=1767%:%
%:%1112=1767%:%
%:%1113=1768%:%
%:%1126=1779%:%
%:%1127=1780%:%
%:%1128=1781%:%
%:%1129=1781%:%
%:%1130=1782%:%
%:%1131=1783%:%
%:%1146=1786%:%
%:%1147=1787%:%
%:%1149=1790%:%
%:%1150=1790%:%
%:%1151=1791%:%
%:%1152=1792%:%
%:%1185=1891%:%
%:%1197=1893%:%
%:%1198=1894%:%
%:%1199=1895%:%
%:%1200=1896%:%
%:%1202=1898%:%
%:%1203=1898%:%
%:%1204=1899%:%
%:%1207=1902%:%
%:%1208=1903%:%
%:%1209=1904%:%
%:%1210=1904%:%
%:%1211=1905%:%
%:%1218=1912%:%
%:%1260=1949%:%
%:%1261=1950%:%
%:%1262=1951%:%
%:%1263=1952%:%
%:%1264=1953%:%
%:%1265=1954%:%
%:%1267=1956%:%
%:%1268=1956%:%
%:%1269=1957%:%
%:%1284=2054%:%
%:%1285=2055%:%
%:%1287=2057%:%
%:%1288=2057%:%
%:%1289=2058%:%